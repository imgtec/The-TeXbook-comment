
The \TeX book

The fine print in the upper right-hand
corner of each page is a draft of intended
index entries; it won't appear in the real book.
Some index entries will be typewriter type t0d0
and/or preceded by \ or .nclosed in <...>, etc;
soch typographic distinctions aren't shown here.
An index entry often extends for several pages;
the actual scope will be determined later.
Please note things that should be indexed but aren't.



  Book printing differs significantly from ordinary typing with respect to
dashes, hyphens, and minus signs. In good math books, these symbols are all
different; in fact there usually are least four different symbols:
    a hyphen (-);
    an en-dash (--);
    an em-dash (---);
    a minus sign ($-$).

   

> EXERCISE 2.1
  Explain how to type the following sentence to \TeX: Alice said, ``I always
use an en-dash instead of a hyphen when specifying page numbers like `480--490'
in a bibliography.''

> EXERCISE 2.2
  What do you think happens when you type four hyphens in a row?
      If you look closely at most well-printed books, you will find that
certain combinations of letters are treated as a unit. For example, this is
true of the `f' and the `i' of `find'. Such combinations 


{Z}
In case you need to type quotes within quotes, for example a single quote
followed by a double quote, you can't simply type '\thinspace'\thinspace' because \TeX quote
interpret this as ''{}'(namely,

{Z}
> EXERCISE 2.4
  OK, now you know how to produce ''\thinspace' and '\thinspace '';
  how do you get ``\thinspace` and `\thinspace``?



\end

